\documentclass[a4j]{jarticle}
\usepackage[dvipdfmx]{graphicx}
\usepackage{here}
\usepackage{ascmac}
\usepackage{url}
\title{
\vspace{30mm}
株式会社マルナカ様\\
購入商品情報管理システム\\
外部設計書v1.0
\vspace{90mm}
}
\author{
株式会社Toron
}

\begin{document}
\maketitle
\newpage
\tableofcontents
\newpage
\section{jsメソッド}
\subsection{ユーザログイン情報取得}
\begin{itemize}
\item メソッド名:AuthentivationUsed\\
Webページでログインする際に、入力されたログイン情報がデータベースに登録されているか問い合わせを行うメソッドである。
引数として、入力されたユーザIDとパスワードを受け取る。データベース
また返り値としてデータベース上にログイン情報が存在したか判定を行った結果を返す。

\item メソッド名:SetJSON\\

JSONファイルを作成するメソッドである。引数は、ユーザIDとパスワード

\item メソッド名:GetAutheticationUser\\

サーバーにJSONファイルを送信するためのメソッドである。
引数はJSONData

\item メソッド名:CheckAutheticationUser\\

サーバ内で該当するユーザ情報が登録されているかJSONファイルをもとに検索するメソッド
多分ウェブじゃない
引数はJSONData
返り値はIdentifivationNumber
この返り値が0の場合、入力されたユーザ情報は該当しない等のエラー
1はユーザ情報が店長
2は管理者

\item メソッド名:ChangePage\\

ページの遷移を行うメソッド
引数はIdentifivationNumber

\subsection{店舗情報登録}
%
\item メソッド名:GetShopData\\

現在登録されている店舗情報をサーバから取得するためのメソッド
引数はなく、返り値はJSONData中の店舗情報等

\item メソッド名:GetShop\\

サーバ内で検索するメソッド
ウェブじゃない

\item メソッド名:CleateTable\\
JSONデータや入力情報をもとにWebページ上で表を作成するメソッドである。\\
引数として、表に書き込むデータを受け取る。
	\begin{itemize}
		\item 引数1:
	\end{itemize}
\item メソッド名:RegistrationShop\\

追加の店舗情報をデータベースに登録するメソッド
引数は店舗IDと店舗名、店長のユーザID
返り値はBoolean Success
この返り値が0の場合、なんらかのエラーで登録ができなかった
1の場合登録成功

\item メソッド名:SetRegistrationShop\\

うぇぶでない

\subsection{店舗情報更新}



\item メソッド名:UpdateShop\\

店舗情報を変更する際の処理を行うメソッド
引数は店舗IDと店舗名、店長のユーザID
返り値はBoolean Success
返り値は0の場合、なんらかのエラーで登録ができなかった
1の場合変更成功

\item メソッド名:SetJSON\\

店舗情報を変更する際に変更情報をサーバに送信するためにJSONファイルを作成するメソッド
引数は店舗IDと店舗名、店長のユーザID
返り値はJSONData

\item メソッド名:UpdateShop\\

サーバーにJSONファイルを送信するためのメソッドである。
引数はJSONData

\item メソッド名:UpdateTable\\

うぇぶでない
\subsection{店舗情報削除}
%
\item メソッド名:DeleteShop\\

店舗情報を削除する際の処理を行うメソッド
引数は店舗IDと店舗名、店長のユーザID
返り値はBoolean Success
返り値は0の場合、なんらかのエラーで登録ができなかった
1の場合変更成功

\item メソッド名:DeleteTable\\

表から削除が完了したデータを削除するメソッド
引数は店舗IDと店舗名、店長のユーザID
返り値なし

\subsection{ログアウト}
%
\item メソッド名:LogOutUser\\
ログアウトボタンを押すと、ログイン前の状態に遷移する動作をさせるメソッド
引数返り値なし
%
\subsection{特売情報登録}
\item メソッド名:GetsaleData\\

登録済みの特売情報をJSONファイルで取得するメソッド
引数なし
返り値はJSONファイル中の表作成に必要なデータと、カテゴリーと商品名

\item メソッド名:GetSale\\

うぇぶじゃない

\item メソッド名:SetChoice\\

カテゴリーと商品名の選択肢を作成するメソッド
引数はJSONDataの必要なデータ
返り値なし

\item メソッド名:SetRegistrationSale\\

特売にする商品のデータベースに登録を行うメソッド
引数は(ないきがするーー)商品ID
返り値はBoolean Success
返り値は0の場合、なんらかのエラーで登録ができなかった
1の場合変更成功

\item メソッド名:SetDeleteSale\\
特売にする商品のデータベースから削除を行うメソッド
引数は(ないきがするーー)商品ID
返り値はBoolean Success
返り値は0の場合、なんらかのエラーで削除ができなかった
1の場合削除成功

\subsection{特価情報登録}
%
\item メソッド名:SetRegistrationSpecialSale\\
特価にする商品のデータベースに登録を行うメソッド
引数は店舗IDと商品IDと割引率と割引フラグ
返り値はBoolean Success
返り値は0の場合、なんらかのエラーで登録ができなかった
1の場合変更成功
%
\item メソッド名:SetDeleteSpecialSale\\
特価にする商品のデータベースに削除を行うメソッド
引数は店舗IDと商品IDと割引率と割引フラグ
返り値はBoolean Success
返り値は0の場合、なんらかのエラーで削除ができなかった
1の場合変更成功
%
\item メソッド名:SetUpdateSpecialeSale
売り切れた特価商品にイベントを起こすメソッドデータベースにフラグもたてる
引数は店舗IDと商品ID
返り値はBoolean Success
返り値は0の場合、なんらかのエラーで削除ができなかった
1の場合変更成功

\end{itemize}

\end{document}