\documentclass[a4j]{jarticle}
\usepackage[dvipdfmx]{graphicx}
\usepackage{here}
\usepackage{ascmac}
\usepackage{url}
\title{
\vspace{30mm}
株式会社マルナカ様\\
購入商品情報管理システム\\
内部設計書v1.0
\vspace{90mm}
}
\author{
株式会社Toron
}

\begin{document}
\maketitle
\newpage
\tableofcontents
\newpage
\section{jsメソッド}
\subsection{ユーザログイン情報取得}
\begin{itemize}
\item メソッド名:AuthentivationUsed\\
Webページでログインする際に、入力されたログイン情報がデータベースに登録されているか問い合わせを行うメソッドである。
引数として、入力されたユーザIDとパスワードを受け取る。
また返り値はである。これは、データベース上にログイン情報が存在したか判定を行った結果を返すものである。

	\begin{itemize}
		\item 引数1:user\_ID
		\item 引数2:password
		\item 返り値:IdentifivationNumber
	\end{itemize}
\item メソッド名:SetJSON\\

JSONファイルを作成するメソッドである。引数は、ユーザIDとパスワードである。生成したJSONファイルを返り値とする。
	\begin{itemize}
		\item 引数1:user\_ID
		\item 引数2:password
		\item 返り値:JSONData
	\end{itemize}
\item メソッド名:GetAutheticationUser\\

サーバーにJSONファイルを送信するためのメソッドである。
引数はJSONDataである。
	\begin{itemize}
		\item 引数1:JSONData
	\end{itemize}
	
%サーバー側
\item メソッド名:CheckAutheticationUser\\

サーバ内で該当するユーザ情報が登録されているかJSONファイルをもとに検索するメソッドである。
引数はJSONDataである。
返り値はIdentifivationNumberである。この変数の値は次のような意味を示す。
	\begin{itemize}
		\item 値が1:ユーザが店長である場合
		\item 値が2:ユーザが管理者である場合
		\item 値が0:入力されたユーザ情報と該当するユーザが存在しないなどのエラー
	\end{itemize}


\item メソッド名:ChangePage\\

引数の値に応じてページの遷移を行うメソッドである。
引数はIdentifivationNumberである。

	\begin{itemize}
		\item 引数1:IdentifivationNumber
	\end{itemize}
\subsection{店舗情報登録}
%
\item メソッド名:GetShopData\\

現在登録されている店舗情報をサーバから取得するためのメソッドである。このメソッドは、webページを開いた時点で呼ばれる。
引数はなく、返り値はJSONDataとなる。
	\begin{itemize}
		\item 引数1:JSONData
	\end{itemize}
%サーバ側
\item メソッド名:GetShop\\

サーバ内で検索する
必要なデータ
・店舗情報
・店長のユーザ情報

\item メソッド名:CleateTable\\
JSONデータや入力情報をもとにWebページ上で表を作成するメソッドである。\\
引数として、JSONDataもしくは、表に書き込む店舗ID、店舗名、店長のユーザIDのデータを受け取る。
	\begin{itemize}
		\item 引数1:JSONData
		
		
		\item 引数1:shop\_ID
		\item 引数2:shopName
		\item 引数3:user\_ID
	\end{itemize}
\item メソッド名:RegistrationShop\\

追加の店舗情報をデータベースに登録するメソッドである。
引数は店舗IDと店舗名、店長のユーザIDである。
返り値はBooleanSuccessである。
	\begin{itemize}
		\item 引数1:shop\_ID
		\item 引数2:shopName
		\item 引数3:user\_ID
		\item 返り値:Boolean Success
	\end{itemize}
また返り値の値の意味は次に示す。
	\begin{itemize}
		\item 値が0:データベースに正しく値が書き込めなかった場合
		\item 値が1:データベース更新成功
	\end{itemize}
%サーバ側	
\item メソッド名:SetRegistrationShop\\


\subsection{店舗情報更新}



\item メソッド名:UpdateShop\\

店舗情報を変更する際の処理を行うメソッドである。
引数は店舗IDと店舗名、店長のユーザIDである。
返り値はBoolean Successである。
	\begin{itemize}
		\item 引数1:shop\_ID
		\item 引数2:shopName
		\item 引数3:user\_ID
		\item 返り値:Boolean Success
	\end{itemize}
また返り値の値の意味は次に示す。
	\begin{itemize}
		\item 値が0:データベースに正しく値が書き換えれなかった場合
		\item 値が1:データベース更新成功
	\end{itemize}

\item メソッド名:SetJSON\\

店舗情報を変更する際に変更情報をサーバに送信するためにJSONファイルを作成するメソッドである。
引数は店舗IDと店舗名、店長のユーザIDである。
返り値はJSONDataである。
	\begin{itemize}
		\item 引数1:shop\_ID
		\item 引数2:shopName
		\item 引数3:user\_ID
		\item 返り値:JSONData
	\end{itemize}
\item メソッド名:UpdateShop\\

サーバーにJSONファイルを送信するためのメソッドである。
引数はJSONDataである。
	\begin{itemize}
		\item 引数1:JSONData
	\end{itemize}
%サーバ側
\item メソッド名:UpdateTable\\

\subsection{店舗情報削除}
%
\item メソッド名:DeleteShop\\

店舗情報を削除する際の処理を行うメソッドである。
引数は店舗IDと店舗名、店長のユーザIDである。
返り値はBoolean Successである。

	\begin{itemize}
		\item 引数1:shop\_ID
		\item 引数2:shopName
		\item 引数3:user\_ID
		\item 返り値:Boolean Success
	\end{itemize}
また返り値の値の意味は次に示す。
	\begin{itemize}
		\item 値が0:データベース上から正しく削除できなかった場合
		\item 値が1:データベースから削除成功
	\end{itemize}

\item メソッド名:DeleteTable\\

表から削除が完了したデータを削除するメソッドである。
引数は店舗IDと店舗名、店長のユーザIDである。
返り値はない。
	\begin{itemize}
		\item 引数1:shop\_ID
		\item 引数2:shopName
		\item 引数3:user\_ID
	\end{itemize}

\subsection{ログアウト}
%
\item メソッド名:LogOutUser\\
ログアウトボタンを押すと、ログイン前の状態に遷移する動作をさせるメソッドである。
引数と返り値はない。
%
\subsection{特売情報登録}
\item メソッド名:GetsaleData\\

登録済みの特売情報をJSONファイルで取得するメソッドである。
引数はない。
返り値はJSONファイルである。
%中の表作成に必要なデータと、カテゴリーと商品名

%サーバ側
\item メソッド名:GetSale\\

\item メソッド名:SetChoice\\

カテゴリーと商品名の選択肢を作成するメソッド
引数はJSONDataであり、JSONファイル中のカテゴリ名と商品名を必要とする。
返り値はない。
	\begin{itemize}
		\item 引数1:JSONData
	\end{itemize}
	
\item メソッド名:SetRegistrationSale\\

特売にする商品のデータベースに登録を行うメソッド
引数は商品IDである%かな?
返り値はBooleanSuccessである。

	\begin{itemize}
		\item 引数1:Registration\_ID
	\end{itemize}
また返り値の値の意味は次に示す。
	\begin{itemize}
		\item 値が0:データベース上から正しく登録できなかった場合
		\item 値が1:データベースに登録成功
	\end{itemize}
\item メソッド名:SetDeleteSale\\
特売にする商品のデータベースから削除を行うメソッドである。
引数は商品IDである。
返り値はBoolean Successである。

	\begin{itemize}
		\item 引数1:Registration\_ID
	\end{itemize}
また返り値の値の意味は次に示す。
	\begin{itemize}
		\item 値が0:データベース上から正しく削除できなかった場合
		\item 値が1:データベースに削除成功
	\end{itemize}
\subsection{特価情報登録}
%
\item メソッド名:SetRegistrationSpecialSale\\
特価にする商品のデータベースに登録を行うメソッドである。
引数は店舗IDと商品IDと割引率と割引フラグである。
返り値はBooleanSuccessである。
	\begin{itemize}
		\item 引数1:shop\_ID
		\item 引数2:Registration\_ID
		\item 引数3:rate
		\item 引数4:rateFlag
	\end{itemize}
また返り値の値の意味は次に示す。
	\begin{itemize}
		\item 値が0:データベース上から正しく登録できなかった場合
		\item 値が1:データベースに登録成功
	\end{itemize}
%
\item メソッド名:SetDeleteSpecialSale\\
特価にする商品のデータベースに削除を行うメソッドである。
引数は店舗IDと商品IDと割引率と割引フラグである。
返り値はBooleanSuccessである。
	\begin{itemize}
		\item 引数1:shop\_ID
		\item 引数2:Registration\_ID
		\item 引数3:rate
		\item 引数4:rateFlag
	\end{itemize}
また返り値の値の意味は次に示す。
	\begin{itemize}
		\item 値が0:データベース上から正しく削除できなかった場合
		\item 値が1:データベースに削除成功
	\end{itemize}
%
\item メソッド名:SetUpdateSpecialeSale
売り切れた特価商品にイベントを起こすメソッドである。この際にデータベースのsoldoutの更新も行う。
引数は店舗IDと商品IDである。
返り値はBooleanSuccessである。
	\begin{itemize}
		\item 引数1:shop\_ID
		\item 引数2:Registration\_ID

	\end{itemize}
また返り値の値の意味は次に示す。
	\begin{itemize}
		\item 値が0:データベース上から正しく更新できなかった場合
		\item 値が1:データベースに更新成功
	\end{itemize}
\end{itemize}

\end{document}