\documentclass[a4j]{jarticle}
\usepackage[dvipdfmx]{graphicx}
\usepackage{here}
\usepackage{ascmac}
\usepackage{url}
\title{
\vspace{30mm}
株式会社マルナカ様\\
購入商品情報管理システム\\
外部設計書v1
\vspace{90mm}
}
\author{
株式会社Toron
}

\begin{document}
\maketitle
\newpage
\tableofcontents
\newpage

\section{データベース設計}
\begin{table}[H]
\caption{購入情報}
\label{tab:DB1}
\begin{center}
\begin{tabular}{|c|c|c|c|c|}
\hline
属性 &データ型&データ長&SQLite&キー:テーブル名\\ \hline\hline
ユーザID &半角英数字型&7文字固定長&TEXT&PK(商品ID、作成日時と複合)、 FK: ユーザ情報 \\ \hline
商品ID &半角英数字型&10文字固定長&TEXT&PK(ユーザID、作成日時と複合) FK: 商品\\ \hline
商品個数 &整数型&2桁固定長&INTEGER&- \\ \hline
店舗ID &半角英数字型&3文字固定長&TEXT&FK:店舗 \\ \hline
購入価格&整数型&5桁固定長&INTEGER&- \\ \hline
作成日時  &日付型&20文字固定長&NUMERIC&PK(商品ID、ユーザIDと複合) \\ \hline
更新日時 &日付型&20文字固定長&NUMERIC&-  \\ \hline
\end{tabular}
\end{center}
\end{table}


\begin{table}[H]
\caption{ユーザ情報}
\label{tab:DB2}
\begin{center}
\begin{tabular}{|c|c|c|c|c|}
\hline
属性 &データ型&データ長&SQLite&キー:テーブル名\\ \hline\hline
ユーザID &半角英数字型&7文字固定長&TEXT&PK  \\ \hline
ログインID &半角英数字型&10文字可変長&TEXT&- \\ \hline
ユーザネーム &全角文字型&15文字可変長&TEXT&- \\ \hline
パスワード &半角英数字型&16文字固定長&TEXT&-  \\ \hline
WAON番号  &半角英数字型&16文字固定長&TEXT&-  \\ \hline
セキュリティーコード &半角英数字型&16文字固定長&TEXT&-  \\ \hline
管理者フラグ &半角英数字型&1文字固定長&TEXT&-  \\ \hline
 店舗ID &半角英数字型&3文字固定長&TEXT&FK:店舗 \\ \hline
ポイント  &整数型&桁固定長&INTEGER&- \\ \hline
 作成日時  &日付型&20文字固定長&NUMERIC&- \\ \hline
更新日時 &日付型&20文字固定長&NUMERIC&-  \\ \hline
\end{tabular}
\end{center}
\end{table}

\begin{table}[H]
\caption{商品}
\label{tab:DB3}
\begin{center}
\begin{tabular}{|c|c|c|c|c|}
\hline
属性 &データ型&データ長&SQLite&キー:テーブル名\\ \hline\hline
商品ID &半角英数字型&10文字固定長&TEXT&PK \\ \hline
商品名&全角文字型&50文字可変長&TEXT&- \\ \hline
値段&整数型&5桁固定長&INTEGER&-\\ \hline
カテゴリID &半角英数字型&2文字固定長&TEXT&FK:カテゴリ \\ \hline
特売フラグ  &半角英数字型&1文字固定長&TEXT&-  \\ \hline
作成日時  &日付型&20文字固定長&NUMERIC&- \\ \hline
更新日時 &日付型&20文字固定長&NUMERIC&-  \\ \hline
\end{tabular}
\end{center}
\end{table}

\begin{table}[H]
\caption{店舗}
\label{tab:DB4}
\begin{center}
\begin{tabular}{|c|c|c|c|c|}
\hline
属性 &データ型&データ長&SQLite&キー:テーブル名\\ \hline\hline
店舗ID &半角英数字型&3文字固定長&TEXT&PK \\ \hline
店舗名 &全角文字型&文字可変長&TEXT& \\ \hline
ユーザID &半角英数字型&7文字固定長&TEXT&FK:ユーザ情報  \\ \hline
作成日時  &日付型&20文字固定長&NUMERIC&- \\ \hline
更新日時 &日付型&20文字固定長&NUMERIC&-  \\ \hline
\end{tabular}
\end{center}
\end{table}

\begin{table}[H]
\caption{保有食品}
\label{tab:DB5}
\begin{center}
\begin{tabular}{|c|c|c|c|c|}
\hline
属性 &データ型&データ長&SQLite&キー:テーブル名\\ \hline\hline
ユーザID &半角英数字型&7文字固定長&TEXT&FK:ユーザ  \\ \hline
商品ID &半角英数字型&10文字固定長&TEXT&FK: 商品\\ \hline
商品個数 &整数型&2桁固定長&INTEGER&- \\ \hline
 作成日時  &日付型&20文字固定長&NUMERIC&- \\ \hline
更新日時 &日付型&20文字固定長&NUMERIC&-  \\ \hline
\end{tabular}
\end{center}
\end{table}



\begin{table}[H]
\caption{カテゴリ}
\label{tab:DB6}
\begin{center}
\begin{tabular}{|c|c|c|c|c|}
\hline
属性 &データ型&データ長&SQLite&キー:テーブル名\\ \hline\hline
カテゴリID &半角英数字型&2文字固定長&TEXT&PK \\ \hline
カテゴリ名&全角文字列型&10文字可変長&TEXT&- \\ \hline
商品ID &半角英数字型&10文字固定長&TEXT&FK: 商品\\ \hline
作成日時  &日付型&20文字固定長&NUMERIC&- \\ \hline
更新日時 &日付型&20文字固定長&NUMERIC& - \\ \hline
\end{tabular}
\end{center}
\end{table}

\begin{table}[H]
\caption{特価}
\label{tab:DB7}
\begin{center}
\begin{tabular}{|c|c|c|c|c|}
\hline
属性 &データ型&データ長&SQLite&キー:テーブル名\\ \hline\hline
商品ID &半角英数字型&10文字固定長&TEXT&PK(店舗IDと複合)、FK: 商品\\ \hline
割引値 &整数型&3桁固定長&INTEGER&- \\ \hline
割引率フラグ &半角英数字型&1文字固定長&TEXT&-  \\ \hline
売り切れフラグ&半角英数字型&1文字固定長&TEXT&-  \\ \hline
店舗ID &半角英数字型&3文字固定長&TEXT&PK(商品IDと複合)、FK:店舗 \\ \hline
 作成日時  &日付型&20文字固定長&NUMERIC&- \\ \hline
更新日時 &日付型&20文字固定長&NUMERIC&-  \\ \hline
\end{tabular}
\end{center}
\end{table}

\begin{table}[H]
\caption{各テーブルに対して可能な操作を示した表}
\label{tab:DB権限}
\begin{center}
\begin{tabular}{|c|c|c|c|c|c|c|c|c|c|c|c|c|c|c|}%操作ができないテーブルがあるのはおかしい%管理者ページと店長ページをまとめて書いて、管理サーバ?か何かたす?
\hline
&&\multicolumn{4}{|c|}{アンドロイドアプリ}&\multicolumn{4}{|c|}{管理者ページ}&\multicolumn{4}{|c|}{レジシステム}\\ \hline
データテーブル&&追加&参照&更新&削除&追加&参照&更新&削除&追加&参照&更新&削除\\ \hline\hline
購入情報	&& &〇&	&	& 	&	&	&	&〇&&&\\ \hline
ユーザ情報	&&〇 &〇&〇	&	&	&〇&〇&〇& &〇&&\\ \hline
商品		&& &〇&	&	& 	&〇&	&	& &〇&&\\ \hline
店舗		&& &〇&	&	&〇&〇&〇&〇& &〇&&\\ \hline
保有食品	&& &〇&〇&〇& 	&	&	&	& 〇&&&\\ \hline
カテゴリ		&& &〇&	&	&	&〇&	&	& &〇&&\\ \hline
特価		&& &〇&	&	&〇&〇&〇&〇& &〇&&\\ \hline

\end{tabular}
\end{center}
\end{table}

\end{document}
